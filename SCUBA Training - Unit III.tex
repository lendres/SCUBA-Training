	% Unit III.
	\chapter*{Unit III}
	\setcounter{questionnumber}{0}

	% 1.
	\begin{qanda}
		\begin{question}
What six general environmental conditions can affect you in any aquatic environment?
		\end{question}

		\begin{answer}
The six general environmental conditions that can affect you are:
			\begin{nospacenumberedlist}
				\item Temperature
				\item Visibility
				\item Water movement
				\item Bottom composition
				\item Aquatic life
				\item Sunlight
			\end{nospacenumberedlist}
		\end{answer}
	\end{qanda}

	% 2.
	\begin{qanda}
		\begin{question}
How can you obtain an orientation to an unfamiliar aquatic environment?
		\end{question}

		\begin{answer}
The PADI Discover Local Diving experience is one way.  Another is to dive under the supervision of an experienced local diver.
		\end{answer}
	\end{qanda}

	% 3.
	\begin{qanda}
		\begin{question}
How can you expect temperature to change with depth?
		\end{question}

		\begin{answer}
It usually gets colder with depth.
		\end{answer}
	\end{qanda}

	% 4.
	\begin{qanda}
		\begin{question}
What's a thermocline?
		\end{question}

		\begin{answer}
A boundary between two distinct layers of water with different temperatures.
		\end{answer}
	\end{qanda}

	% 5.
	\begin{qanda}
		\begin{question}
How should you plan to dive in an area known to have a thermocline?
		\end{question}

		\begin{answer}
Plan the dive for the colder water.  Ask a local for information about potential thermoclines and if you encounter one unexpectedly consider staying shallower in the warm water.
		\end{answer}
	\end{qanda}

	% 6.
	\begin{qanda}
		\begin{question}
What's the definition of ``underwater visibility?''
		\end{question}

		\begin{answer}
It is based on how far you can see horizontally.
		\end{answer}
	\end{qanda}

	% 7.
	\begin{qanda}
		\begin{question}
What four principle factors affect underwater visibility?
		\end{question}

		\begin{answer}
The factors affecting underwater visibility are:
			\begin{nospacenumberedlist}
				\item Water movement
				\item Weather
				\item Suspended particles
				\item Bottom composition
			\end{nospacenumberedlist}
		\end{answer}
	\end{qanda}

	% 8.
	\begin{qanda}
		\begin{question}
Restricted visibility can affect you in what three ways?
		\end{question}

		\begin{answer}
Restricted visibility can affect you in the following ways:
			\begin{nospacebulletedlist}
				\item It's more difficult to stay with your buddy
				\item It's harder to keep track of where you are and where you're going
				\item You may feel disoriented when you can't see the surface or bottom for reference
			\end{nospacebulletedlist}%
		\end{answer}%
	\end{qanda}%

%\item You may feel disoriented when you can't see the surface or bottom for reference asdf asdf asdf asdf asdf%

	% 9.
	\begin{qanda}
		\begin{question}
How do you avoid the problems associated with diving in clear water?
		\end{question}

		\begin{answer}
Watch your depth gauge and stay within your planned depth limit.  A reference point to look at will help prevent disorientation.
		\end{answer}
	\end{qanda}

	% 10.
	\begin{qanda}
		\begin{question}
What four primary causes generate surface and underwater currents?
		\end{question}

		\begin{answer}
The four primary causes of currents are:
			\begin{nospacenumberedlist}
				\item Winds blowing over the surface
				\item Unequal heating and cooling of water
				\item Tides
				\item Waves
			\end{nospacenumberedlist}
		\end{answer}
	\end{qanda}

	% 11.
	\begin{qanda}
		\begin{question}
What should you do if you get caught in a current and carried downstream past a predetermined destination or exit point?
		\end{question}

		\begin{answer}
Swim across the current where you may be able to reach a line trailed from the boat or reach shore.
		\end{answer}
	\end{qanda}

	% 12.
	\begin{qanda}
		\begin{question}
In most circumstances, which way should you go when there's a mild current present?
		\end{question}

		\begin{answer}
Begin your dive by slowly swimming into the current.  Swim against currents on the bottom where they are usually less strong.
		\end{answer}
	\end{qanda}

	% 13.
	\begin{qanda}
		\begin{question}
What should you do if you get exhausted and caught in a current at the surface while diving from a boat?
		\end{question}

		\begin{answer}
Don't fight it.  Fill your BCD to establish buoyancy, signal for help, rest and wait for the boat to pick you up.
		\end{answer}
	\end{qanda}

	% 14.
	\begin{qanda}
		\begin{question}
Aquatic bottom compositions include what six types?
		\end{question}

		\begin{answer}
The aquatic bottom compositions are:
			\begin{nospacenumberedlist}
				\item Silt
				\item Mud
				\item Sand
				\item Rock
				\item Coral
				\item Vegetation
			\end{nospacenumberedlist}
		\end{answer}
	\end{qanda}

	% 15.
	\begin{qanda}
		\begin{question}
What are the two ways to avoid bottom contact?
		\end{question}

		\begin{answer}
Use effective buoyancy control, secure equipment, and stay well off the bottom.  Swim with your fins up to avoid stirring the sediment and reducing visibility.
		\end{answer}
	\end{qanda}

	% 16.
	\begin{qanda}
		\begin{question}
What are the two basic classifications for interaction between divers and aquatic live?
		\end{question}

		\begin{answer}
Passive and active.
		\end{answer}
	\end{qanda}

	% 17.
	\begin{qanda}
		\begin{question}
What causes nearly all injuries from aquatic life?
		\end{question}

		\begin{answer}
Defensive actions take by the animal.
		\end{answer}
	\end{qanda}

	% 18.
	\begin{qanda}
		\begin{question}
What should you do if you sight an aggressive animal underwater?
		\end{question}

		\begin{answer}
Remain still and calm on the bottom, watch the animal, but don't swim towards it.  Swim away if it remains in the area.
		\end{answer}
	\end{qanda}

	% 19.
	\begin{qanda}
		\begin{question}
Nine simple precautions minimize the likelihood of being injured by an aquatic animal.  What are they?
		\end{question}

		\begin{answer}
The precautions that minimize the likelihood of injury are:
			\begin{nospacenumberedlist}
				\item Treat all animals with respect
				\item Be cautious in extremely murky water where you may have trouble watching where you put your hands
				\item Avoid wearing shiny dangling jewelry
				\item If you spearfish, remove speared fish from the water immediately
				\item Wear gloves and an exposure suit to avoid stings and cuts
				\item Maintain neutral buoyancy and stay off the bottom
				\item Move slowly and carefully
				\item Watch where you're going and where you put your hands, feet, and knees
				\item Avoid contact with unfamiliar animals
			\end{nospacenumberedlist}
		\end{answer}
	\end{qanda}

	% 20.
	\begin{qanda}
		\begin{question}
Why should divers follow local fish and game laws?
		\end{question}

		\begin{answer}
The laws exist to assure a continuing supply of these animals for the future.
		\end{answer}
	\end{qanda}

	% 21.
	\begin{qanda}
		\begin{question}
How can you prevent sunburn while out of the water (three ways), and what two ways can you use to prevent it while snorkeling?
		\end{question}

		\begin{answer}
You can prevent sunburn by the following.

\noindent Out of the water:
			\begin{numberedlist}
				\item Wear protective clothing
				\item Stay in the shade as much as possible
				\item Use sunscreen
			\end{numberedlist}

\noindent While snorkeling:
			\begin{nospacenumberedlist}
				\item Wear an exposure suit
				\item Wear waterproof sunscreen
			\end{nospacenumberedlist}
		\end{answer}
	\end{qanda}

	% 22.
	\begin{qanda}
		\begin{question}
What are the general considerations for diving in freshwater, and in saltwater?
		\end{question}

		\begin{answer}
Freshwater:
			\begin{bulletedlist}
				\item Currents
				\item Bottom compositions
				\item Limited visibility
				\item Thermoclines
				\item Cold water
				\item Entanglement
				\item Deep water
				\item Boats
				\item High altitude
			\end{bulletedlist}
Saltwater:
			\begin{nospacebulletedlist}
				\item The same as freshwater
				\item Waves
				\item Surf
				\item Tides
				\item Currents
				\item Coral
				\item Marine life
				\item Remote locations
			\end{nospacebulletedlist}
		\end{answer}
	\end{qanda}

	% 23.
	\begin{qanda}
		\begin{question}
What creates surge and how do you avoid it?
		\end{question}

		\begin{answer}
Waves passing overhead cause surge.  Planning a deeper dive can help you avoid surge.
		\end{answer}
	\end{qanda}

	% 24.
	\begin{qanda}
		\begin{question}
What causes long shore currents, and how may they affect you?
		\end{question}

		\begin{answer}
Waves hitting the shore at an angle.  They may push you down the shore, so start your dive into the current so you can drift back to the exit at the end of dive.
		\end{answer}
	\end{qanda}

	% 25.
	\begin{qanda}
		\begin{question}
Why would a wave break offshore?
		\end{question}

		\begin{answer}
An offshore reef wreck or sand bar can create a shallow area that causes waves to break.
		\end{answer}
	\end{qanda}

	% 26.
	\begin{qanda}
		\begin{question}
What causes a rip current, and how do you know when there's one present?
		\end{question}

		\begin{answer}
A rip current occurs when waves push water over a long obstruction such as a sand bar or reef.  The water can't flow out on the bottom, so it funnels back to sea through a narrow opening.
		\end{answer}
	\end{qanda}

	% 27.
	\begin{qanda}
		\begin{question}
What should you do if you get caught in a rip current?
		\end{question}

		\begin{answer}
Establish buoyancy and swim parallel to the shore to clear the rip area.
		\end{answer}
	\end{qanda}

	% 28.
	\begin{qanda}
		\begin{question}
What causes an upwelling, and how might it affect local offshore dive conditions?
		\end{question}

		\begin{answer}
An upwelling is a slow-moving current commonly caused by offshore winds pushing the surface water away from shore.  It moves water up from bottom which can make water clear and colder.
		\end{answer}
	\end{qanda}

	% 29.
	\begin{qanda}
		\begin{question}
Tidal movement changes what three environmental conditions?
		\end{question}

		\begin{answer}
Currents, depth, and visibility.
		\end{answer}
	\end{qanda}

	% 30.
	\begin{qanda}
		\begin{question}
What's generally the best tidal level for diving?
		\end{question}

		\begin{answer}
High tide.
		\end{answer}
	\end{qanda}

	% 31.
	\begin{qanda}
		\begin{question}
You need to plan your dive for what three reasons?
		\end{question}

		\begin{answer}
The three reasons to plan a dive are:
			\begin{nospacenumberedlist}
				\item Avoids misunderstandings with your buddy
				\item Avoids forgotten equipment
				\item Avoids poor dive conditions
			\end{nospacenumberedlist}
		\end{answer}
	\end{qanda}

	% 32.
	\begin{qanda}
		\begin{question}
What are the four stages of proper dive planning?
		\end{question}

		\begin{answer}
The four stages of dive planning are:
			\begin{nospacenumberedlist}
				\item Advance planning
				\item Preparation
				\item Last-minute preparation
				\item Pre-dive planning
			\end{nospacenumberedlist}
		\end{answer}
	\end{qanda}

	% 33.
	\begin{qanda}
		\begin{question}
What five general steps do you follow during the advanced planning stage of dive planning?
		\end{question}

		\begin{answer}
The five general steps of the advanced diving stage are:
			\begin{nospacenumberedlist}
				\item Get a buddy
				\item Establish a dive objective
				\item Choose a dive site
				\item Determine best time to dive
				\item Discuss logistics (when/where to meet)
			\end{nospacenumberedlist}
		\end{answer}
	\end{qanda}

	% 34.
	\begin{qanda}
		\begin{question}
What four general steps do you follow during the preparation stage of dive planning?
		\end{question}

		\begin{answer}
The four steps of the preparation stage are:
			\begin{nospacenumberedlist}
				\item Inspect equipment
				\item Fill tank
				\item Gather equipment
				\item Use equipment checklist to be sure you have everything
			\end{nospacenumberedlist}
		\end{answer}
	\end{qanda}

	% 35.
	\begin{qanda}
		\begin{question}
What five steps do you follow during the last-minute preparation stage of dive planning?
		\end{question}

		\begin{answer}
The five steps of the last-minute preparation stage are:
			\begin{nospacenumberedlist}
				\item Check the weather
				\item Let someone who isn't going with you know about your planned dive
				\item Gather those last-minute type items like a jacket, hat, et cetera
				\item If you haven't yet, pack your gear bag
				\item Make an ``idiot check'' so that you don't leave anything behind
			\end{nospacenumberedlist}
		\end{answer}
	\end{qanda}

	% 36.
	\begin{qanda}
		\begin{question}
What seven steps do you follow during the pre-dive planning stage of dive planning?
		\end{question}

		\begin{answer}
The seven pre-dive steps are:
			\begin{nospacenumberedlist}
				\item Evaluate the conditions
				\item Decide whether or not conditions favor the dive and your objective
				\item Agree on where to enter, the general course to follow, the techniques to use on the dive and where to exit
				\item Review hand signals and other communications
				\item Decide what to do if you become separated
				\item Agree on time, depth, and air supply limits
				\item Discuss what to do if an emergency arise
			\end{nospacenumberedlist}
		\end{answer}
	\end{qanda}

	% 37.
	\begin{qanda}
		\begin{question}
What are three benefits of diving from a boat?
		\end{question}

		\begin{answer}
The three benefits of boat diving are:
			\begin{nospacenumberedlist}
				\item It eliminates long, tiresome surface swims
				\item Dealing with surf
				\item It eliminates long hikes to and from the water
			\end{nospacenumberedlist}
		\end{answer}
	\end{qanda}

	% 38.
	\begin{qanda}
		\begin{question}
When preparing for a boat dive, what five general considerations apply to equipment preparation?
		\end{question}

		\begin{answer}
Equipment preparation for a boat dive includes:
			\begin{nospacenumberedlist}
				\item Inspect your equipment for potential problems, fill you tank and pack spare parts
				\item Be sure you've marked your stuff so it doesn't get confused with someone else's
				\item Use a dive bag for carrying your equipment to and from the boat
				\item Pack your equipment so what you need first ends up on top
				\item Take ample warm/dry clothing
			\end{nospacenumberedlist}
		\end{answer}
	\end{qanda}

	% 39.
	\begin{qanda}
		\begin{question}
Before a boat dive, what four general considerations for personal preparation apply?
		\end{question}

		\begin{answer}
You should:
			\begin{nospacenumberedlist}
				\item Be well rested
				\item Avoid excessive alcohol the night before
				\item Avoid foods you don't digest well
				\item Be well hydrated
			\end{nospacenumberedlist}
		\end{answer}
	\end{qanda}

	% 40.
	\begin{qanda}
		\begin{question}
What part of the boat is:
			\begin{nospacenumberedlist}
				\item Bow (forward)
				\item Stern (aft)
				\item Starboard
				\item Port
				\item Leeward
				\item Windward
				\item Bridge
				\item Head
				\item Galley
			\end{nospacenumberedlist}
		\end{question}

		\begin{answer}
They are:
			\begin{nospacenumberedlist}
				\item Front
				\item Back
				\item Right
				\item Left
				\item Side opposite of the wind
				\item Side wind is coming from
				\item Cabin with the boat controls
				\item Bathroom
				\item Kitchen
			\end{nospacenumberedlist}
		\end{answer}
	\end{qanda}

	% 41.
	\begin{qanda}
		\begin{question}
By what four ways can you minimize the effects of motion sickness while on a boat?
		\end{question}

		\begin{answer}
You can minimize motion sickness by:
			\begin{nospacebulletedlist}
						\item Taking seasickness medication
						\item Avoid greasy foods
						\item Say in fresh air on the deck
						\item Stay in the center of boat and watch the horizon
			\end{nospacebulletedlist}
		\end{answer}
	\end{qanda}

	% 42.
	\begin{qanda}
		\begin{question}
By what three ways can you prevent or control most dive problems that occur at the surface?
		\end{question}

		\begin{answer}
Prevention methods include:
			\begin{nospacenumberedlist}
				\item Staying with in your limitations
				\item Relaxing
				\item Establishing and maintaining buoyancy
			\end{nospacenumberedlist}
		\end{answer}
	\end{qanda}

	% 43.
	\begin{qanda}
		\begin{question}
What should you do if a diving-related problem occurs at the surface?
		\end{question}

		\begin{answer}
Immediately establish buoyancy by inflating your BCD or dropping your weights.  Get help when you need it, before a small problem becomes a big one.
		\end{answer}
	\end{qanda}

	% 44.
	\begin{qanda}
		\begin{question}
How do the appearance and actions of a diver who is under control differ from the appearance and actions of a diver who has, or is about to have, a problem involving panic?
		\end{question}

		\begin{answer}
If they are under control they will mostly look like a diver without problems, they are relaxed and breathe normally, keep their equipment in place, move with controlled, deliberate movements, and respond to instructions.  A panics diver loses self control and sudden, unreasoned fear and instinctive inappropriate actions replace controlled, appropriate actions.
		\end{answer}
	\end{qanda}

	% 45.
	\begin{qanda}
		\begin{question}
What are the four basic steps to assisting another diver?
		\end{question}

		\begin{answer}
The four basic steps to assisting another diver are:
			\begin{nospacenumberedlist}
				\item Establish ample buoyancy for both of you
				\item Calm the diver
				\item Help the diver reestablish breathing control
				\item If necessary, assist the diver back to the boat or shore
			\end{nospacenumberedlist}
		\end{answer}
	\end{qanda}

	% 46.
	\begin{qanda}
		\begin{question}
By what three ways can you prevent or control most dive problems that may occur underwater?
		\end{question}

		\begin{answer}
You can prevent or control most underwater problems by:
			\begin{nospacenumberedlist}
				\item Relaxing while you dive
				\item Keeping close watch on your air supply
				\item Diving within your limitations
			\end{nospacenumberedlist}
		\end{answer}
	\end{qanda}

	% 47.
	\begin{qanda}
		\begin{question}
What are four problems that may occur underwater?
		\end{question}

		\begin{answer}
Problems that may occur under water include:
			\begin{nospacenumberedlist}
				\item Overexertion
				\item Running out of or low on air
				\item Regulator free flow
				\item Entanglement
			\end{nospacenumberedlist}
		\end{answer}
	\end{qanda}

	% 48.
	\begin{qanda}
		\begin{question}
What, in order of priority, are the five low-on-air/out-of-air emergency procedures?
		\end{question}

		\begin{answer}
The low-on-air/out-of-air emergency procedures are:
			\begin{nospacenumberedlist}
				\item Make a normal ascent
				\item Ascend using an alternate air source
				\item Ascend using a controlled emergency swimming ascent
				\item Buddy breathe with a single regulator
				\item Make a buoyant emergency ascent
			\end{nospacenumberedlist}
		\end{answer}
	\end{qanda}

	% 49.
	\begin{qanda}
		\begin{question}
How do you breathe from a free-flowing regulator?
		\end{question}

		\begin{answer}
Hold the regulator in your hand and press the mouthpiece to the outside of your lips, inserting one corner if you like.  Breathe the air you need like drinking from a water fountain, letting the excess air escape.
		\end{answer}
	\end{qanda}

	% 50.
	\begin{qanda}
		\begin{question}
What should you do if you become entangled underwater?
		\end{question}

		\begin{answer}
Stop and work slowly and calmly to free yourself.  Don't twist or turn because this may wrap the line around you.
		\end{answer}
	\end{qanda}

	% 51.
	\begin{qanda}
		\begin{question}
What are the four general procedures for dealing with an unresponsive diver in the water.
		\end{question}

		\begin{answer}
The general procedures for dealing with an unresponsive diver in the water are:
			\begin{nospacenumberedlist}
				\item Quickly bring the diver to the surface and check for breathing
				\item Establish ample positive buoyancy for you and the victim
				\item Get assistance as needed in providing rescue breathing
				\item Help remove the diver from the water
			\end{nospacenumberedlist}
		\end{answer}
	\end{qanda} 