% Unit II.
    \chapter*{Unit I}

    % 1.
    \begin{qanda}
\question{What will the buoyancy of an object be (positive, neutral, or negative) if it displaces an amount of water:
    \begin{bulletedlist}
        \item more than its own weight?
        \item less than its own weight?
        \item equal to its own weight?
    \end{bulletedlist}
    \vspace*{-\baselineskip}}

\answer{The buoyancy of the object will be:
    \begin{bulletedlist}
        \item Positive
        \item Negative
        \item Neutral
    \end{bulletedlist}}
    \end{qanda}

    % 2.
    \begin{qanda}
\question{Why is buoyancy control, both at the surface and underwater, one of the most important skills a diver can master?}

\answer{It lets you control where you are in the water.  At the surface, positive buoyancy lets you rest and save energy.  Underwater, you'll remain neutrally buoyant most of the time so you can swim effortlessly and move freely in all directions and stay off of the bottom.}
    \end{qanda}

    % 3.
    \begin{qanda}
\question{What two items control a diver's buoyancy?}

\answer{Lead weights and a buoyancy control device (BDC).}
    \end{qanda}

    % 4.
    \begin{qanda}
\question{How does the buoyancy of an object differ in fresh water compared to salt water?}

\answer{Objects are more buoyant in salt water because the dissolved salts increase the density of the water.}
    \end{qanda}

    % 5.
    \begin{qanda}
\question{How does lung volume affect buoyancy?}

\answer{Exhaling decreases buoyancy and inhaling increases buoyancy.}
    \end{qanda}

    % 6.
    \begin{qanda}
\question{Why do you usually only feel changing pressure in your body's air spaces?}

\answer{Your body is primarily liquid, which is incompressible and distributes pressure equally throughout your entire body.  Air is compressible so the air spaces in your body change volume unless the pressure is equalized.}
    \end{qanda}

    % 7.
    \begin{qanda}
\question{Why are pressure changes while ascending or descending underwater much more substantial than pressure changes when ascending or descending the same distance in air?}

\answer{Because water is denser than air, pressure changes more rapidly for a given distance ascent or descent.}
    \end{qanda}

    % 8.
    \begin{qanda}
\question{What is the relationship between increasing and decreasing depth and water pressure?}

\answer{Every ten meters of water adds one atmosphere.}
    \end{qanda}

    % 9.
    \begin{qanda}
\question{What are the absolute pressures, in atmospheres or bar, for:
    \begin{bulletedlist}
        \item 10 meters / 33 feet?
        \item 20 meters / 66 feet?
        \item 30 meters / 99 feet?
        \item 40 meters / 132 feet?
    \end{bulletedlist}
    \vspace*{-\baselineskip}}

\answer{The absolute pressures are:
    \begin{bulletedlist}
        \item 1 atm
        \item 2 atm
        \item 3 atm
        \item 4 atm
    \end{bulletedlist}}
    \end{qanda}

    % 10.
    \begin{qanda}
\question{What is the relationship between air volume and density, and how do they vary according to this relationship when pressure increases or decreases?}

\answer{The volume is inversely proportional (1:1) to pressure.  $Density = \frac{Mass}{Volume}$, so density is proportional to pressure.  $V_d = \frac{V_i}{D/10 + 1}$.}
    \end{qanda}

    % 11.
    \begin{qanda}
\question{What are the three major air spaces affected by pressure?}

\answer{The two major air spaces within your body most noticeably affected by increasing pressure are your ears and sinuses.  The major artificial air space most affected by increasing pressure is the one created by your mask.}
    \end{qanda}

    % 12.
    \begin{qanda}
\question{What is a ``squeeze?''}

\answer{Squeeze is when  the water pressure compresses the air in your body air spaces and the volume decreases, pushing body tissues in.}
    \end{qanda}

    % 13.
    \begin{qanda}
\question{What is ``equalization?''}

\answer{Keeping the volume in air spaces normal by adding air to it during descent, which keeps the air space pressure equal to the water pressure outside.}
    \end{qanda}

    % 14.
    \begin{qanda}
\question{What are three ways you can equalize air spaces during descent?}

\answer{Your ear and the sinus air spaces connect to the throat, allowing you to use air from your lungs to equalize them.  You equalize the air space in your mask through your nose.  To equalize the air space in your ears, pinch your nose shut and gently blow against it with your mouth closed; this directs air from your throat into your ears and sinus air spaces.  Another technique is swallowing and wiggling the jaw from side to side.  A third technique combines these -- swallowing and wiggle your jaw while blowing gently against your pinched nose.}
    \end{qanda}

    % 15.
    \begin{qanda}
\question{How often should you equalize during descent?}

\answer{Every meter.}
    \end{qanda}

    % 16.
    \begin{qanda}
\question{What three steps should you take if you feel discomfort in an air space while descending?}

\answer{Ascend until the discomfort eases, equalize and continue a slow descent equalizing more frequently.}
    \end{qanda}

    % 17.
    \begin{qanda}
\question{What is the most important rule in scuba diving?}

\answer{The most important rule in scuba diving is to breathe continuously and never, never hold your breath.}
    \end{qanda}

    % 18.
    \begin{qanda}
\question{What are the consequences of breaking the most important rule in scuba diving?}

\answer{Lung over pressurization, will occur unless you permit the pressure to equalize by breathing normally at all times.  Lung over expansion can force air into the bloodstream and chest cavity, which can lead to severe injuries including paralysis and death.}
    \end{qanda}

    % 19.
    \begin{qanda}
\question{What is a ``reverse block?''}

\answer{A reverse block occurs when expanding air cannot escape from an air space during ascent.}
    \end{qanda}

    % 20.
    \begin{qanda}
\question{What should you do if you feel discomfort during ascent due to air expansion in the ears, sinuses, stomach, intestines, or teeth?}

\answer{Slow or stop your ascent, descend a meter and give the trapped air time to work its way out.}
    \end{qanda}

    % 21.
    \begin{qanda}
\question{How does increasing depth affect how long your air supply lasts?}

\answer{You air supply does not last as long at increasing depths because the air supplied by the SCUBA equipment is denser.}
    \end{qanda}

    % 22.
    \begin{qanda}
\question{What's the most efficient way to breathe dense air underwater?}

\answer{Take deep slow breaths.}
    \end{qanda}

    % 23.
    \begin{qanda}
\question{Why does a diver need a mask?}

\answer{You need a mask to see underwater because light behaves differently in water.}
    \end{qanda}

    % 24.
    \begin{qanda}
\question{Why does the mask need to enclose your nose?}

\answer{The mask creates an air space that must be equalized during descent.  The enclosed nose provides a means to equalize the pressure.}
    \end{qanda}

    % 25.
    \begin{qanda}
\question{What six features should you look for in a mask?}

\answer{The six features are:
    \begin{numberedlist}
        \item Tempered-glass lens plate
        \item Comfortable skirt with a close fit to your face
        \item Nose or finger pockets to make equalizing your ears easier
        \item Low-profile.  Lower-file means less air is required to equalize and a wider vision field
        \item Adjustable strap that can be locked in place
        \item Wide field of vision.  Low-profile or wrap around designs give this
    \end{numberedlist}}
    \end{qanda}

    % 26.
    \begin{qanda}
\question{When buying a mask, what are the two most important factors?}

\answer{Fit and comfort.}
    \end{qanda}

    % 27.
    \begin{qanda}
\question{How do you prepare a new mask for use?}

\answer{Remove the film from manufacturing using a soft cloth and non-gel toothpaste or other low abrasion cleaner with fine grit that can remove the film without scratching the glass.}
    \end{qanda}

    % 28.
    \begin{qanda}
\question{What three general maintenance procedures apply to mask care?}

\answer{Rinse thoroughly with fresh water after each use (even in a swimming pool), keep out of direct sunlight, and store in a cool, dry place.}
    \end{qanda}

    % 29.
    \begin{qanda}
\question{Why does a diver need a snorkel?}

\answer{It lets you rest or swim with your face in the water, like when you're looking for something below, without wasting cylinder air.  When there's a bit of surface chop, splashing waves can get in your mouth if you don't have a snorkel, but the snorkel is usually high enough to get above these.  If you run low on air away from the boat or shore, it makes it easier to swim back, again resting with your face in the water.}
    \end{qanda}

    % 30.
    \begin{qanda}
\question{What three features does an easy-breathing snorkel have?}

\answer{The three features are:
    \begin{numberedlist}
        \item A large bore
        \item Not excessively long -- if it's too long it's hard to clear
        \item Designed with smooth, rounded bends
    \end{numberedlist}}
    \end{qanda}

    % 31.
    \begin{qanda}
\question{When purchasing a snorkel, how do you check it for fit and comfort?}

\answer{Place the snorkel in your mouth with the mouthpiece flange between your lips and teeth, and the barrel of the snorkel against your left ear.  You should be able to adjust the mouthpiece to fit comfortably, without chaffing or causing jaw fatigue, while sitting straight in your mouth.}
    \end{qanda}

    % 32.
    \begin{qanda}
\question{How do you prepare a new snorkel for use?}

\answer{Attach the snorkel to the left side of your mouth so the top of the snorkel sits at the crown of your head.}
    \end{qanda}

    % 33.
    \begin{qanda}
\question{Why does a diver need fins?}

\answer{They help you move more effectively by letting you use your leg muscles to swim.}
    \end{qanda}

    % 34.
    \begin{qanda}
\question{What are the two basic fin styles?}

\answer{Adjustable strap and full-foot.}
    \end{qanda}

    % 35.
    \begin{qanda}
\question{What blade design features may enhance a fin's performance?}

\answer{Ribs, vents, channels, and split fins are all features which may enhance fin's performance.}
    \end{qanda}

    % 36.
    \begin{qanda}
\question{How do you prepare new fins for use?}

\answer{Adjust the straps for a snug, comfortable fit with your wet-suit boots on.}
    \end{qanda}

    % 37.
    \begin{qanda}
\question{What three considerations do you have when buying a specific type of fin?}

\answer{Your size, physical ability, and where you intend to use them.  Your primary concerns are fit and comfort.}
    \end{qanda}

    % 38.
    \begin{qanda}
\question{Why does a diver need a BCD?}

\answer{To regulate your buoyancy while underwater and for providing positive buoyancy for resting, swimming, or lending assistance to others.}
    \end{qanda}

    % 39.
    \begin{qanda}
\question{Why do divers need a backpack?}

\answer{To hold the cylinder on your back.}
    \end{qanda}

    % 40.
    \begin{qanda}
\question{Of the three styles of BCD, which is the most commonly used by recreational divers?}

\answer{Jacket-style.}
    \end{qanda}

    % 41.
    \begin{qanda}
\question{What five features do BCDs have in common?}

\answer{The five features are:
    \begin{numberedlist}
         \item It must hold enough air to give you and your equipment ample buoyancy at the surface
         \item It must have a large-diameter inflation/deflation hose so you can release air quickly and easily
         \item It should have a low-pressure inflation system that fills your BCD with air directly from your cylinder
         \item It must have an over pressure relief valve to prevent the BCD from rupturing due to overfilling or due to air expansion during ascent
         \item It should be adjustable enough to fit comfortably and not ride up on your body when you inflate it
    \end{numberedlist}}
    \end{qanda}

    % 42.
    \begin{qanda}
\question{How do you prepare a BCD for use?}

\answer{Adjust the fit.  Too loose and it will rotate around you, too tight and it will restrict breathing.}
    \end{qanda}

    % 43.
    \begin{qanda}
\question{What two special maintenance procedures apply to caring for a BCD?}

\answer{You need to rinse the inside as well as the outside with fresh water.  To do this fill it about one third with water through the inflator hose, the the rest of the way with air.  Swish the water around and drain.  Store the BCD partially inflated.  This keeps the bladder from sticking together internally.}
    \end{qanda}

    % 44.
    \begin{qanda}
\question{Why does a diver need a scuba cylinder?}

\answer{To safely store high-pressure air so you have something to breathe underwater.}
    \end{qanda}

    % 45.
    \begin{qanda}
\question{What does a cylinder valve do?}

\answer{To control air flow from the cylinder.}
    \end{qanda}

    % 46.
    \begin{qanda}
\question{With what piece of equipment is the backpack usually integrated?}

\answer{The BCD.}
    \end{qanda}

    % 47.
    \begin{qanda}
\question{What are the four common sizes and the two materials for scuba cylinders?}

\answer{8, 10, 12, and 15 liters (SI); 50 71.2, and 80 ft\supscript{3} (imperial).  Cylinders are made of aluminum or steel.}
    \end{qanda}

    % 48.
    \begin{qanda}
\question{What five markings do you commonly find on the neck of a scuba cylinder?}

\answer{The five markings are:
    \begin{numberedlist}
        \item The material the cylinder is made of
        \item The maximum permitted pressure
        \item Serial number
        \item Dates of all pressure tests
        \item Manufacturer or distributor symbol
    \end{numberedlist}}
    \end{qanda}

    % 49.
    \begin{qanda}
\question{What are the two basic types of cylinder valves?}

\answer{K-valve (a simple on/off valve) and a J-valve (has a built-in mechanism that signals when you run low on air).}
    \end{qanda}

    % 50.
    \begin{qanda}
\question{What does a J-valve do, and why is its use declining?}

\answer{The J-valve contains a spring-operated shutoff valve that is held open by cylinder pressure until the pressure drops to approximately 20-40 bar/300-500 psi.  They are not used as much anymore because they are prone to tripping (so they don't warn you), they increase the cost and service requirements of the valve, and the functionality has been replaced by the more reliable submersible pressure gauge (SPG).}
    \end{qanda}

    % 51.
    \begin{qanda}
\question{What's the difference between a DIN valve and a yoke valve?}

\answer{The yoke valve attaches the regulator via a yoke assembly.  You can easily identify a yoke valve by the o-ring seal on the front of the valve and dimpled guide for the yoke on the back.  The DIN valve system screws the regulator into the valve.  The DIN system is rated for higher pressures.}
    \end{qanda}

    % 52.
    \begin{qanda}
\question{What is the purpose of a burst disc?}

\answer{To relieve cylinder over pressurization.}
    \end{qanda}

    % 53.
    \begin{qanda}
\question{What three safety precautions for handling scuba cylinders should you follow going to and at a dive site?}

\answer{The safety precautions are:
    \begin{bulletedlist}
        \item Don't leave them standing unattended
        \item If they need to be standing, be sure they are secured so they can't fall
        \item When carrying your cylinders in your car, lay them down horizontally and block or tie them so they cannot roll
    \end{bulletedlist}}
    \end{qanda}

    % 54.
    \begin{qanda}
\question{How do you turn a cylinder valve on and off?}

\answer{Open the valve slowly, all the way until it stops turning.  Close valves gently and avoid over tightening.}
    \end{qanda}

    % 55.
    \begin{qanda}
\question{What's the best way to keep water out of a scuba cylinder?}

\answer{Never allow it to completely empty.}
    \end{qanda}

    % 56.
    \begin{qanda}
\question{Why do you need scuba cylinder visual inspections and pressure tests?}

\answer{The inspections check for rust and corrosion, the pressure test checks for metal fatigue.}
    \end{qanda}

    % 57.
    \begin{qanda}
\question{What does a regulator do?}

\answer{The regulator reduces the scuba cylinder's high pressure air to match the surrounding water pressure, and it delivers air only on demand.}
    \end{qanda}

    % 58.
    \begin{qanda}
\question{When looking at a regulator, which are the following parts:
    \begin{bulletedlist}
        \item first stage?
        \item second stages?
        \item dust cover?
        \item purge button?
    \end{bulletedlist}
    \vspace*{-\baselineskip}}

\answer{The regulator parts are:
    \begin{bulletedlist}
        \item The first stage is the chrome and brass piece that connects to the cylinder
        \item The second stages contain the mouth pieces and purge button
        \item The dust cover is attached to the first stage
        \item The purge button is located on the second stage
    \end{bulletedlist}}
    \end{qanda}

    % 59.
    \begin{qanda}
\question{What's the most important feature for consideration when purchasing a regulator?}

\answer{Ease of breathing.}
    \end{qanda}

    % 60.
    \begin{qanda}
\question{How do you rinse a regulator after use, and what three points do you need to keep in mind while doing so?}

\answer{When rinsing the regulator, keep in mind:
    \begin{numberedlist}
        \item Put the first stage dust cover firmly in place to keep water out of the first stage
        \item Do not use high-pressure water to rinse your regulator -- only gently flowing water
        \item Don't press the purge button while rinsing or soaking, because this opens the second stage inlet valve and can allow water to flow up the hose into the first stage
    \end{numberedlist}}
    \end{qanda}

    % 61.
    \begin{qanda}
\question{Why do divers need a submersible pressure gauge (SPG)?}

\answer{The SPG tells you how much air you have during a dive.}
    \end{qanda}

    % 62.
    \begin{qanda}
\question{What are the three reasons for diving with a buddy at all times?}

\answer{The three reasons are:
    \begin{numberedlist}
        \item Practicality
        \item Safety
        \item Fun
    \end{numberedlist}}
    \end{qanda} 